\documentclass[a4paper,11pt]{article}

\usepackage[utf8]{inputenc}
\usepackage[T1]{fontenc}
\usepackage{siunitx}
\usepackage[german]{babel} 
\selectlanguage{german}

\title{Der Baum-Adapter}
\author{Matthias Bock}

\begin{document}

\maketitle

\section{Kurzbeschreibung}

Der Baum-Adapter ist eine Leiterplatte,
welche die LEDs an der Haupts\"aule im Baumhaus ansteuert.
Der Adapter beinhaltet einen Sockel
f\"ur den Anschluss unserer Mikrocontroller-Platine.
Ein externes Netzteil ist zur Spannungsversorgung
von Adapter, Mikrocontroller und LEDs erforderlich.

\newpage
\section{Kontext}

\subsection{Adressierbare LEDs}
An der Haupts\"aule ("Baum") im Baumhaus wurden
16 untereinander unverbundene Streifen
von jeweils ca. 2,5m L\"ange angebracht.
Auf den Streifen befinden sich pro Meter
60 adressierbare LEDs vom Typ SK6812-WWA.
Jede LED kann in drei verschiedenen Farben leuchten:
warmwei{\ss}, kaltwei{\ss} und bernstein.
Alle drei Farben sind in 256 Intensit\"atsabstufungen einstellbar (8 Bit).
Die insgesamt 24 Bit erwarten die LEDs
in einem Takt von ca. 800kHz
pulsweitencodiert
an ihrem Dateneingang.
Nachfolgende Bits werden am Datenausgang der LEDs
an die jeweils nachfolgende LED im Streifen weitergereicht.

\subsection{Adapter}
Es ist derzeit vorgesehen,
dass insgesamt vier Baum-Adapter
alle 16 LED-Streifen steuern,
wobei ein Adapter die jeweils vier LED-Streifen
einer Seite der S\"aule ansteuert.
Der Adapter ist mit den LED-Streifen
\"uber Stromversorgungs- und Datenkabel verbunden.

\subsection{Mikrocontroller}
Die Steuerung der LEDs \"ubernimmt die
Mikrocontroller-Platine Waveshare Core51822,
die von der \"ubergeordneten Steuerungseinheit
(z.B. Raspberry Pi)
angesprochen werden kann.
Der darauf verbaute Mikrocontroller nRF51822 von Nordic Semiconductor koordiniert die auf den LEDs ablaufenden Muster
durch die Berechnung aller jeweils aktuellen
LED-Leuchtintensit\"atswerte zu einem Zeitpunkt (Frame)
und deren regelm\"a{\ss}ige Ausgabe an die LED-Streifen.
Die Ausgabe erfolgt
mit einer bestimmten Wiederholrate (Framerate).

\subsection{Montage}
Die Adapter werden in Aufputzdosen
auf der S\"aule befestigt.
Die Aufputzdosen bestehen aus selbstl\"oschendem Polyvinylchlorid
mit einer nominalen Temperaturbest\"andigkeit bis 60\si{\degree}C
sowie einer Flammenwidrigkeit bis 850\si{\degree}C.

\subsection{Stromversorgung}
Adapter und LEDs jeder S\"aulenseite
werden von jeweils einem, gemeinsamen
Netzteil versorgt.

\newpage
\section{Anforderungen}

\subsection{Gr\"o{\ss}e}
Der Adapter, mit allen Komponenten aufmontiert,
muss in den Abmessungen einer Aufputzdose Platz finden.

\subsection{Funktionsumfang}
\subsubsection{LED-Steuerung}
Der Adapter muss die M\"oglichkeit implementieren,
vier Ausgangssignale des Mikrocontrollers
mit den Eing\"angen von vier LED-Streifen zu verbinden.

\subsubsection{Pegelwandlung}
Der Adapter muss die LED-Steuersignale des Mikrocontrollers
von 3.3V auf 5V anheben.

\subsubsection{Debugging}
Der Adapter muss eine M\"oglichkeit zum Anschluss
eines Debuggers implementieren,
sodass der Mikrocontroller
im fertig verbauten Aufputzgeh\"ause
durch \"Offnen derselben und ohne L\"oten
programmiert und gedebuggt werden kann.

\subsubsection{Standby-Power}
Der Adapter muss den Mikrocontroller
von der Standby-Spannung
des Netzteils mit 3.3V versorgen.

\subsubsection{Netzteilsteuerung}
Der Adapter muss den Mikrocontroller in die Lage versetzen,
das Netzteil an und aus zu schalten.

\subsubsection{Stromschalter}
Der Adapter muss eine M\"oglichkeit implementieren,
die Stromzuf\"uhrung zu den LEDs
bei angeschaltetem Netzteil zu unterbrechen.

\newpage
\subsection{Elektrische Sicherheit}
\subsubsection{Verhindern von \"Ubertemperatur}
Es m\"ussen Ma{\ss}nahmen getroffen werden, die sicherstellen,
dass der Adapter,
die darauf best\"uckten Komponenten
und der Mikrocontroller
sich nicht derart erhitzen k\"onnen,
dass sie Feuer fangen oder die Gefahr besteht,
dass das Plastik der Aufputzdose
schmilzt oder Feuer f\"angt.

\subsubsection{Detektion \"uberm\"a{\ss}iger Stromaufnahme der LEDs}
Der Adapter muss in der Lage sein, zu detektieren und zu reagieren,
wenn die angeschlossenen LEDs mehr Strom aufnehmen,
als bei maximaler Helligkeit zu erwarten w\"are.

\subsubsection{Detektion \"uberm\"a{\ss}iger Stromaufnahme vom Netzteil}
Der Adapter muss in der Lage sein, zu detektieren und zu reagieren,
wenn dem angeschlossenen Netzteil mehr Leistung entnommen wird,
als das Netzteil seiner Spezifikation entsprechend maximal liefern kann.


\newpage
\section{Umsetzung}

\end{document}
