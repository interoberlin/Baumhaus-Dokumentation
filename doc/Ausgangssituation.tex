
\section{Ausgangssituation}

\subsection{Adressierbare LEDs}
An der Haupts\"aule ("Baum") im Baumhaus wurden
16 LED-Streifen
von jeweils ca. 2,5m L\"ange angebracht.
Auf den Streifen befinden sich pro Meter
60 adressierbare LEDs vom Typ SK6812-WWA.
Jede LED kann in drei verschiedenen Farben leuchten:
warmwei{\ss}, kaltwei{\ss} und bernstein.
Alle drei Farben sind in jeweils
256 Intensit\"atsabstufungen einstellbar (8 Bit pro Farbe).
Die LEDs erwarten ihre Konfigurationsbytes
mit einem Takt von ca. 800kHz
pulsweitencodiert
am Dateneingang des LED-Streifens
(1 Dateneingang pro Streifen).

\subsection{Adapter}
Es ist derzeit vorgesehen,
dass insgesamt vier Baum-Adapter
die 16 LED-Streifen steuern,
wobei ein Adapter die jeweils vier LED-Streifen
einer Seite der S\"aule ansteuert.
Der Adapter ist mit den LED-Streifen
\"uber Stromversorgungs- und Datenkabel verbunden.

\subsection{Mikrocontroller}
Die Steuerung der LEDs \"ubernimmt die
Mikrocontroller-Platine
Waveshare Core51822\footnote{\url{http://www.waveshare.com/wiki/Core51822}},
die von der \"ubergeordneten Steuerungseinheit
(u.a. Raspberry Pi)
angesprochen wird.
Der darauf verbaute Mikrocontroller
nRF51822\footnote{\url{https://www.nordicsemi.com/eng/Products/Bluetooth-low-energy/nRF51822}}
von Nordic Semiconductor
koordiniert die auf den LEDs ablaufenden Muster
durch die Berechnung aller jeweils aktuellen
LED-Leuchtintensit\"atswerte zu einem Zeitpunkt (Frame)
und deren regelm\"a{\ss}ige Ausgabe an die LED-Streifen.
Die Ausgabe erfolgt
mit einer bestimmten Wiederholrate (Framerate).

\subsection{Montage}
Die Adapter werden in Aufputzdosen
auf der S\"aule befestigt.
Aus den Aufputzdosen werden nach Bedarf
Kabel zu Netzteil und LEDs herausgef\"uhrt.

\subsection{Stromversorgung}
Adapter und LEDs jeder S\"aulenseite
werden von jeweils einem, gemeinsamen
Netzteil versorgt.

