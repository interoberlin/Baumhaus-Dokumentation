
\section{Auswirkungen}

Nachfolgend werden die Auswirkungen der oben beschriebenen
Umsetzung des Adapters auf das Gesamtsystem beschrieben.

\subsection{Stromaufnahmemessung}

Die Firmware muss
in regelm\"a{\ss}igen Intervallen
eine Messung der
Stromaufnahme der LEDs
(siehe \ref{Stromaufnahmemessung})
vornehmen.
Das Ergebnis soll in einer Variable
f\"ur die interne
und an einer BLE-Charakteristik f\"ur externe
Weiterverarbeitung bereit stehen.

\subsection{Kurzschlussdetektion}

Wird bei der Stromaufnahmemessung
ein Kurzschlussstrom detektiert,
muss die Firmware die Anlage abschalten
und \"uber eine BLE-Charakteristik
ein entsprechendes Fehlersignal ausgeben.

Ein Kurzschlussstrom sei eine Stromst\"arke,
die um wenigstens 5\% h\"oher ausf\"allt,
als bei maximaler Helligkeit
aller angeschlossenen LEDs
zu erwarten w\"are.

\subsection{\"Uberstromdetektion}

Die Firmware soll
in regelm\"a{\ss}igen Intervallen
aus den eingeschalteten LEDs
den erwarteten Stromverbrauch errechnen
und mit der gemessenen Stromaufnahme vergleichen.
Ist der gemessene Strom
um mehr als 10\% der maximalen Stromaufnahme
einer S\"aulenseite
h\"oher,
als der erwartete,
so soll die Firmware die Anlage abschalten
und \"uber eine BLE-Charakteristik
ein entsprechendes Fehlersignal ausgeben.

\subsection{\"Uberlastungsdetektion}

Die Firmware muss
in regelm\"a{\ss}igen Intervallen
aus der Stromaufnahmemessung
die dem Netzteil entnommene Leistung berechnen
und mit der einprogrammierten,
maximalen Leistungsabgabe des Netzteils vergleichen.

Wird die Maximalleistung des Netzteils \"uberschritten,
so soll die Firmware die Anlage abschalten
und \"uber eine BLE-Charakteristik
ein entsprechendes Fehlersignal ausgeben.
