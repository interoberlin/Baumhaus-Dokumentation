
\section{Umsetzung}
\subsection{Pegelwandlung}
Der 3,3V-Logikpegel wird auf 5V gewandelt
mithilfe des
\begin{center}
	Texas Instruments SN74LVC1G125DBVR\footnote{\url{http://www.ti.com/lit/ds/symlink/sn74lvc1g125.pdf}}
\end{center}
Dieser IC ben\"otigt
zum ordnungsgem\"a{\ss}en Betrieb
einen niederinduktiv platzierten St\"utzkondensator
mit etwa 10-100nF
auf dem 5V-Pegel.

\subsection{Mikrocontroller-Sockel mit Debugger-Buchse}
Zum Einsatz kommen,
wie gehabt\footnote{\url{https://github.com/interoberlin/nRF51-Boards/}},
zwei Buchsenleisten mit jeweils 2x9 Pins im Rasterma{\ss} 2mm
f\"ur die Mikrocontroller-Platine,
sowie eine 2x5 Pin IDC-Buchse
mit AVR-Debugger-Pinbelegung.

\subsection{Kupferfreie Fl\"ache unter der Antenne}
Der Bereich auf dem Adapter-PCB,
\"uber dem auf der Mikrocontroller-Platine die Antenne zu liegen kommt,
muss auf beiden Seiten von Kupfer befreit werden,
um nicht die Abstimmung der RF-Anordnung zu beeinflussen.

\subsection{Standby-Power}
Ein 3,3V-LDO versorgt den Mikrocontroller
aus der 5V-Standby-Versorgung des Netzteils.
Sobald das Netzteil angeschaltet ist,
wird der Mikrocontroller
aus dessen 3,3V-Ausgang gespei{\ss}t.

\subsection{Netzteil-Schalter}
Der N-FET 2N7002\footnote{\url{https://www.nxp.com/documents/data_sheet/2N7002.pdf}}
verbindet die Einschaltleitung des ATX-Netzteils
mit Masse.
Sein Gate
ist mit einem Ausgang
des Mikrocontrollers
verbunden
und mit einem $20k\Omega$-Pull-Down-Widerstand versehen
zur Abschaltung des Netzteils
im Falle einer Fehlfunktion.

Ein $20k\Omega$-Pull-Up-Widerstand auf der Einschaltleitung
stellt sicher, dass ohne Gate-Signal am FET
kein Einschaltsignal gegeben wird.

\subsection{Stromaufnahmemessung}
\label{Stromaufnahmemessung}
Die LEDs erhalten vom Netzteil 5V.
Die Stromaufnahme
wird zur Laufzeit kontinuierlich
gemessen.

\begin{center}
Allegro Microsystems ACS712-30A\footnote{\url{http://www.allegromicro.com/~/media/files/datasheets/acs712-datasheet.ashx}}
\end{center}
Das resultierende Messsignal liegt
in analoger Form
an jeweils einem Pin des Mikrocontrollers
zur Auswertung
mithilfe des nRF51822-internen A/D-Wandlers
an.

\subsection{Leistungsschalter}
Ein N-FET schaltet die R\"uckf\"uhrung (Minuspol)
der 5V-Spannung
von den LEDs
zur Netzteil-Masse durch.
Die Gates sind mit Ausg\"angen des Mikrocontrollers verbunden
und mit einem $20k\Omega$-Pull-Down-Widerstand versehen
zum Sperren des FETs
-im Falle einer Fehlfunktion.

Das Package ist
aus thermischen Gr\"unden
so gew\"ahlt,
dass der FET auf der Oberseite des PCB
zu liegen kommt.

\subsection{Temperaturmessung}
Die Temperatur des Adapters wird an zwei Stellen verfolgt.
Zum Einen verf\"ugt der nRF51822 \"uber einen internen Temperatursensor.
Zum Anderen ist auf dem Adapter ein
\begin{center}
	Dallas Semiconductor DS18B20\footnote{\url{http://datasheets.maximintegrated.com/en/ds/DS18B20.pdf}}
\end{center}
verbaut,
dessen Ein-/Ausgang
mit dem Mikrocontroller verbunden ist.

\subsection{\"Uberstromschutzeinrichtung}
Eine Schmelzsicherung verhindert
eine \"uberm\"a{\ss}ige Stromaufnahme
der LEDs.
Zum Einsatz kommt eine KFZ-Flachstecksicherung
mit einem Schmelzstrom zwischen 25A und 30A.

Als Sockel k\"onnen KFZ-Sicherungshalter oder Flachsteckh\"ulsen zum Einsatz kommen.
