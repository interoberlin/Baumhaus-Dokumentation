
\subsection{Kabeldimensionierung}
Adapter und LEDs sind über Kabel verbunden.
Diese sind
hinsichtlich ihres Kupferquerschnitts
der Stromaufnahme entsprechend
dimensioniert:

Pro S\"aulenseite werden maximal 120W aufgenommen
(siehe \ref{Stromaufnahme-LEDs})
.
Die Leistung wird an drei Stellen in die LED-Streifen eingespeist.
An jeder Einspeisestelle
werden entsprechend
$
\frac{120W}{3} = 40W
$
eingespeist.

Am Fu{\ss} der S\"aule werden 5V
bei maximal
$
\frac{40 W}{3} = 13.3 W
$
eingespeist.
Zur Mitte und zum Kopf werden 12V geleitet,
die ein DC/DC-Wandler
auf 5V
herunterkonvertiert.
Durch die h\"ohere Spannung reduziert sich
der zum Erreichen der einzuspeisenden Leistung
erforderliche Stromfluss.
Der Stromfluss ist f\"ur den
Spannungsabfall
im Kabel
ma{\ss}geblich:
\begin{center}
	$
	P = {\Delta}U \cdot I = R \cdot I^2
	$
\end{center}

Der mindestens erforderliche
Querschnitt A des Kabels
errechnet sich nach
\footnote{
	\url{http://www.elektro-fachplanung.de/Fachinfo/Planungshilfen/Berechnung\_01/berechnung\_01.htm}
}
:

\begin{center}
	$
	A = \frac{I \cdot 2 \cdot l}{K \cdot U \cdot {\Delta}u}
	$
\end{center}

Es flie{\ss}t ein Strom von
$
I = 6,67 A
$
auf einer L\"ange von
$
l = 3m
$
bzw.
der doppelten L\"ange (Hin- und R\"uckleiter)
.

Der Leitwert von Kupfer betr\"agt
$
K = 56 \frac{\Omega \cdot mm^2}m
$
.

Es liegt eine Gleichspannung von
$
U = 12V
$
an.

Der akzeptable Spannungsabfall betrage ${\Delta}U = 6\%$.

Daraus ergibt sich ein erforderlicher Querschnitt
\footnote{
	\url{https://www.dieleitungsberechnung.de/?page\_id=124}
}
von
$
1.5mm^2
$
.
